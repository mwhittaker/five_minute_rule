\documentclass[12pt]{article}

\title{The Five-Minute Rule}
\author{Michael Whittaker}
\date{}

\usepackage[margin=1in]{geometry}
\usepackage{color}
\usepackage{graphicx}
\usepackage{mathtools}
\usepackage{soul}
\usepackage{subcaption}

% https://brand.berkeley.edu/colors/
\definecolor{berkblue}   {RGB}{4,   30,  66}
\definecolor{berkgold}   {RGB}{255, 199, 44}
\definecolor{berkbluealt}{RGB}{79,  117, 139}
\definecolor{berkgoldalt}{RGB}{209, 144, 0}

\DeclarePairedDelimiter{\ceil}{\lceil}{\rceil}
\newcommand{\emf}[1]{\textcolor{berkgoldalt}{#1}}
\newcommand{\figref}[1]{Figure~\ref{fig:#1}}
\newcommand{\secref}[1]{Section~\ref{sec:#1}}

\begin{document}
\maketitle

\begin{abstract}
  Disk bandwidth costs money. RAM costs money. If I have a piece of memory, does
  it cost less money to store it on disk or in RAM? Jim Gray, Franco Putzolu,
  and Goetz Graefe initially answered this question with the \emph{five-minute
  rule} in 1987~\cite{gray19875} and then again ten~\cite{gray1998five} and
  twenty~\cite{graefe2007five} years later. In this paper, we explain the
  five-minute rule in plain English (\secref{therule}) and then use it to make
  some pretty graphs (\secref{graphs}).
\end{abstract}

\section{The Five-Minute Rule}\label{sec:therule}
\newcommand{\pagesize}{\textsc{PageSize}}
\newcommand{\ramprice}{\textsc{RamPrice}}
\newcommand{\accessprice}{\textsc{AccessPrice}}
\newcommand{\memsize}{\textsc{MemSize}}
\newcommand{\accessperiod}{\textsc{AccessPeriod}}

A disk is a big long piece of \st{spaghetti} memory segmented into pages. The
disk provides an interface that allows you to read it one page at a time. No
more, no less. You want to read less than a page? Too bad. You can't. How big
are the pages? They could be 4096 bytes or 1024 bytes or 42 bytes; it doesn't
much matter to us. From now on, we'll denote the number of bytes in a page as
\emf{\pagesize{}} (e.g. 4096, 2048, 42).

When it comes to disks, faster is better. That is, we want our disks to be able
to read as many pages per second as possible. Unfortunately, speed ain't cheap;
we'll have to pay for throughput. A disk that can read 100 pages per second
will cost more than one that can only read 1 page per second. We'll assume each
page per second of throughput costs some fixed amount \emf{\accessprice} (e.g.
\$2000).

Like disk, RAM is a big long piece of \st{spaghetti} memory. Unlike disk, RAM
isn't divided into pages. We can read and write as much or as little data into
RAM as we'd like. Want to read 4096 bytes from RAM? No problem. How about 2048
bytes? Fine by me. 42 bytes? Go ahead! Moreover, when it comes to RAM, bigger
is better. Unfortunately, size ain't cheap; we'll have to pay for every byte of
RAM. Let's assume that each byte of RAM costs some fixed amount
\emf{\ramprice{}} (e.g. \$5).

Now imagine we have some object we'd like to store somewhere. Every so often,
we read the object. For example, maybe the object is some 4 byte integer that
we read every second. Or, maybe it's a 10 MB movie that we ready every 6 hours.
Let \emf{\memsize{}} be the size of the object in bytes (e.g. 4, 10485760), and
let \emf{\accessperiod{}} be the number of seconds between each access (e.g. 1,
21600).

We can store this object on disk or in RAM. Let's look at how much both of
these alternatives are going to cost us.

\newcommand{\numpages}{\ceil{\frac{\memsize}{\pagesize}}}
\begin{itemize}
  \item \emph{On Disk.}
    If our object is \memsize{} bytes big and there are \pagesize{} bytes per
    page, then we'll have to store the object on $\numpages$ pages. If we
    access the object every \accessperiod{} seconds, then reading the object
    requires $\frac{\numpages}{\accessperiod}$ pages per second of bandwidth.
    With each page per second of bandwidth costing \accessprice{} dollars,
    storing the object on disk is going to cost, in dollars:
    \[
      \frac{\numpages}{\accessperiod} \times \accessprice
    \]
  \item \emph{In RAM.}
    Our object is \memsize{} bytes and each byte of RAM costs \ramprice{}
    dollars, so storing the object in RAM costs, in dollars:
    \[
      \ramprice \times \memsize
    \]
\end{itemize}

We want to store an object in RAM whenever it's cheaper to do so. That is, when
\[
  \frac{\numpages}{\accessperiod} \times \accessprice
  \geq
  \ramprice \times \memsize
\]
Solving for \accessperiod{}, we get:
\[
  \accessperiod
  \leq
  \frac{\numpages \times \accessprice}{\ramprice \times \memsize}
\]
In words, given a disk with \pagesize{} bytes per page and \accessprice{}
dollars per page per second of bandwidth, a RAM with \ramprice{} dollars per
byte, and a memory object that's \memsize{} bytes big, we'll store the object
in RAM whenever we access it every
$\frac{\numpages \times \accessprice}{\ramprice \times \memsize}$
seconds or fewer.

Gray et al. substituted 1987 era values for \pagesize{}, \accessprice{},
\ramprice{}, and \memsize{} and found \accessperiod{} to be around five
minutes. Hence, the five-minute rule.

\section{Some Pretty Graphs}\label{sec:graphs}
We can use the five-minute rule formula to generate some nice graphs. Letting
\begin{itemize}
  \item \pagesize{} = 4 KB,
  \item \accessprice{} = $\frac{\$2000}{page/second}$, and
  \item \ramprice{} = $\frac{\$5}{KB}$,
\end{itemize}
we can graph the cost of storing objects of various sizes with various access
periods on disk and in RAM. This is shown in \figref{costs}. The point where
the disk cost and RAM cost intersect is called the \emph{break even point}. If
the object is accessed less frequently than the break even point, then it
should be stored in RAM; otherwise, it should be stored on disk.

Furthermore, given disks and RAMs with various values of \pagesize{},
\accessprice{}, and \ramprice{}, we can graph the break even point for objects
of various sizes. This is given in \figref{break-even}.

\begin{figure}
  \centering

  \newcommand{\costs}[2]{
    \begin{subfigure}[b]{0.32\textwidth}
      \centering
      \includegraphics[width=\textwidth]{costs-#1KB.pdf}
      \caption{#2}
      \label{fig:costs-#1}
    \end{subfigure}%
  }

  \costs{01}{1 KB}
  \costs{02}{2 KB}
  \costs{03}{3 KB}

  \costs{04}{4 KB}
  \costs{05}{5 KB}
  \costs{06}{6 KB}

  \costs{07}{7 KB}
  \costs{08}{8 KB}

  \caption{Disk and RAM costs}
  \label{fig:costs}
\end{figure}

\begin{figure}
  \centering
  \includegraphics[width=\textwidth]{break-even.pdf}
  \caption{Break even points.}
  \label{fig:break-even}
\end{figure}

\bibliographystyle{plain}
\bibliography{bib}
\end{document}
